% This is samplepaper.tex, a sample chapter demonstrating the
% LLNCS macro package for Springer Computer Science proceedings;
% Version 2.20 of 2017/10/04
%
\documentclass[runningheads]{llncs}
\setcounter{secnumdepth}{5}
%
\usepackage{graphicx}
% Used for displaying a sample figure. If possible, figure files should
% be included in EPS format.
%
% If you use the hyperref package, please uncomment the following line
% to display URLs in blue roman font according to Springer's eBook style:
% \renewcommand\UrlFont{\color{blue}\rmfamily}

\usepackage{todonotes}
\usepackage{xcolor}
\usepackage{listings}
\usepackage{mdframed}
\usepackage{caption}

%\usepackage{todo}

\usepackage{array}
\newcolumntype{H}{>{\setbox0=\hbox\bgroup}c<{\egroup}@{}}

\begin{document}
%
\title{Saving Energy Using the READEX Methodology}
%
%\titlerunning{Abbreviated paper title}
% If the paper title is too long for the running head, you can set
% an abbreviated paper title here
%
\author{Madhura Kumaraswamy \inst{1} \and Anamika Chowdhury \inst{1} \and Andreas Gocht \inst{2} \and Jan Zapletal \inst{5} \and Kai Diethelm \inst{7} \and Lubomir Riha \inst{5} \and Marie-Christine Sawley \inst{6} \and Michael Gerndt \inst{1} \and Nico Reissmann \inst{3} \and Ondrej Vysocky \inst{5} \and Othman Bouizi \inst{6} \and Per Gunnar Kjeldsberg \inst{3}  \and Ramon Carreras \inst{4} \and Robert Sch\"one \inst{2} \and Umbreen Sabir Mian \inst{2} \and Venkatesh Kannan \inst{4} \and Wolfgang E. Nagel \inst{2}}
%
\authorrunning{Kumaraswamy, M et al.}
% First names are abbreviated in the running head.
% If there are more than two authors, 'et al.' is used.
%
\institute{ Department of Informatics, Technical University of Munich, Bavaria, Germany 
	\email{kumarasw@in.tum.de}\\ \and
	Technische Universit\"at Dresden, Germany \\ \and
	Norwegian University of Science and Technology, NTNU, Trondheim, Norway \\ \and 
	Irish Centre for High-End Computing, Ireland \\ \and
	IT4Innovations National Supercomputing Center, V\v{S}B -- Technical University of~Ostrava, Ostrava, Czech Republic \\ \and 
	Intel ExaScale Labs, Intel Corp, Paris, France \\ \and
	Gesellschaft f\"ur numerische Simulation GmbH, Germany }

\maketitle              % typeset the header of the contribution

\begin{abstract}
With today's top supercomputers consuming several mega\-watts of power, optimization of energy consumption has become one of the major challenges on the road to exascale computing.
The EU Horizon 2020 project READEX provides a tools-aided auto-tuning methodology to dynamically tune HPC applications for energy-efficiency. READEX is a two-step methodology, consisting of the design-time analysis and runtime tuning stages. At design-time, READEX exploits application dynamism using the \textit{readex\_intraphase} and the \textit{readex\_interphase} tuning plugins, which perform tuning steps, and provide tuning advice in the form of a tuning model. During production runs, the runtime tuning stage reads the tuning model and dynamically switches the settings of the tuning parameters for different application regions. Additionally, READEX also includes a tuning model visualizer and support for tuning application level tuning parameters to improve the result beyond the automatic version. This paper describes the state of the art used in READEX for energy-efficiency auto-tuning for HPC. Energy savings achieved for different proxy benchmarks and production level applications on the Haswell and Broadwell processors highlight the effectiveness of this methodology.

\keywords{DVFS \and energy-efficiency \and application dynamism \and machine learning \and HPC}
\end{abstract}

\section{Introduction} \label{sec:introduction}

The top ranked system in the November 2018 Top500 list is Summit at Oak Ridge National Laboratory. It consumes 9.8 MW with a peak performance of 200 PFlop/s. To reach the exascale level with this technology would already require 48 MW power. Therefore, energy reduction is a major goal for hardware, OS, runtime system, application and tool developers. 

The European Horizon 2020 project READEX (Runtime Exploitation of Application Dynamism for Energy-efficient eXascale Computing)\footnote{www.readex.eu} funded from 2015-2018 developed the READEX tool suite\footnote{We use the abbreviation READEX for the tool suite in the rest of the paper.} for dynamic energy-efficiency tuning of applications. It semi-automatically tunes hybrid HPC applications by splitting the tuning into a \textit{Design-Time Analysis (DTA)} and a \textit{Runtime Application Tuning (RAT)} phase. 

READEX configures different tuning knobs (hardware, software and application parameters) based on dynamic application characteristics. Clock frequencies have a significant impact on performance and power consumption of a processor and therefore to the energy efficiency of the system. To leverage this potential, Intel Haswell processors allow to change core and the uncore frequencies. One basic idea is to lower the core frequency and increase the uncore frequency for memory bound regions since the processor is anyway waiting for data from memory. Another important tuning parameter is the number of parallel OpenMP threads in a node. If a routine is memory bound, it might be more energy-efficient to use only so many threads until the memory bandwidth is saturated~\cite{rs_hotpower12}.

Following the scenario-based tuning approach~\cite{filippopoulos2013exploration} from the embedded world, READEX creates an \textit{Application Tuning Model (ATM)} at design-time which specifies optimal configurations of the tuning parameters for individual program regions. This tuning model is then passed to RAT and is applied during production runs of the application by switching the tuning parameters to their optimal values when a tuned region is encountered. This dynamic tuning approach of READEX allows to exploit dynamic changes in the application characteristics for energy tuning while static approaches only optimize the settings for the entire program run. 

READEX even goes beyond tuning individual regions. It can distinguish different instances of regions, e.g., resulting from different call sites in the code. Such an instance is called \textit{runtime situation (rts)}. Different rts's can have different optimal configurations in the ATM. 

READEX leverages well established tools, such as Score-P for instrumentation and monitoring of the application~\cite{knupfer2012score}. The \textit{READEX Runtime Library} (RRL) implements RAT and is a new plugin for Score-P. The DTA is implemented via new tuning plugins for the Periscope Tuning Framework~\cite{PTF2.0IEEE2016}.  

For reading energy measurements and for modifying core and uncore frequencies, several interfaces are supported, such as libMSRsafe~\cite{msrsave} and LIKWID~\cite{LIKWID}. Power measurements are based on the RAPL~\cite{Intel2018} counters or on special hardware, such as HDEEM~\cite{hdeem} on the Taurus machine in Dresden. If the target machine provides such special hardware, new measurement plugins for Score-P have to be implemented. Moreover, reading energy measurements as well as setting the frequencies can require root privileges and appropriate extensions to the kernel of the machine. The availability of these interfaces is currently the major limitation for applying the READEX tool suite. 

This paper focuses on the features of READEX, the steps to be taken by the user in applying the tools, and motivates why certain aspects are more or less suited for different applications. The paper is neither a user's guide nor a technical documentation of the inner workings of the tools comprising the tool suite. 

Section~\ref{sec:related-work} presents related available tools for energy management of HPC systems. Section~\ref{sec:methodology} introduces the major steps in using the READEX tool suite and presents Pathway \cite{Pathway:Petkov13}, a tool for automating tool workflows for HPC systems. Section~\ref{sec:dta} presents DTA as the first step in tuning applications. The reasoning behind the generation of the ATM at the end of DTA is given in Section~\ref{tm-generation}. Section~\ref{sec:tm-visualization} introduces a tool for the visual analysis of the generated ATM. RAT is presented in Section~\ref{rat}, and the visualization of dynamic switching based on the given ATM in Vampir is introduced in Section~\ref{switching-visualization}. Extensions to the READEX tool suite are presented in Section~\ref{sec:extensions}, and results from several applications on different machines are summarized in Section~\ref{sec:results}. Finally, the paper draws some conclusions. 


\section{State-of-the-Art} \label{sec:related-work}
As energy-efficiency and consumption have now become one of the biggest challenges in HPC, research in this direction has gained momentum. Currently, there are many approaches that employ DVFS in HPC to tune different objectives. Rojek et al.~\cite{Rojek} used DVFS to reduce the energy consumption for stencil computations whose execution time is predefined. The algorithm first collects the execution time and energy for a subset of the processor frequencies, and dynamically models the execution time and the average energy consumption as functions of the operational frequency. It then adjusts the frequency so that the predefined execution time is respected.

Imes et al.~\cite{Imes} developed machine learning classifiers to tune socket resource allocation, HyperThreads and DVFS to minimize energy. The classifiers predict different system settings using the measurements obtained by polling performance counters during the application run. This approach cannot be used for dynamic applications because of the overhead from the classifier in predicting new settings due to rapid fluctuations in the performance counters. READEX, however, focuses on auto-tuning dynamic applications.
	
The ANTAREX project~\cite{silvano2016antarex} specifies adaptivity strategies, parallelization and mapping of the application at runtime by using a Domain Specific Language (DSL) approach. During design-time, control code is added to the application to provide runtime monitoring strategies. At runtime, the framework performs auto-tuning by configuring software knobs for application regions using the runtime information of the application execution. This approach is specialized for ARM-based multi-cores and accelerators, while READEX targets all HPC systems.

Intel's open-source Global Extensible Open Power Manager (GEOPM)~\cite{geopm} runtime framework provides a plugin based energy management system for power-constrained systems. GEOPM supports both offline and online analysis by sampling performance counters, identifying the nodes on the critical path, estimating the power consumption and then adjusting the power budget among these nodes. GEOPM adjusts individual power caps for the nodes instead of a uniform power capping by allocating a larger portion of the job power budget to the critical path. 
	
\textit{Conductor}~\cite{Marathe}, a runtime system used at the Lawrence Livermore National Laboratory also performs adaptive power balancing for power capped systems. It first performs a parallel exploration of the configuration space by setting a different thread concurrency level and DVFS configuration on each MPI process, and statically selects the optimal configuration for each task. It then performs adaptive power balancing to distribute more power to the critical path.

The AutoTune project~\cite{guillen2016dvfs,AutoTune:Book2015} developed a DVFS tuning plugin to auto-tune energy consumption, total cost of ownership, energy delay product and power capping. The tuning is performed using a model that predicts the energy and power consumption as well as the execution time at different CPU frequencies. It uses the enopt library to vary the frequency for different application regions. While this is a static approach, READEX implements dynamic tuning for rts's.

READEX goes beyond previous works by tuning the uncore frequency. It also exploits the dynamism that exists between individual iterations of the main progress loop.


\section{The READEX Methodology} \label{sec:methodology}

The READEX methodolgy is a tools-aided approach in which the READEX tool suite is applied on a HPC application for automatic energy and performance tuning. The steps to apply the READEX tool suite, as illustrated in Fig. \ref{fig:readex_methodology}, on an application are split into two phases: Design time (during application development) and Runtime (during production runs).

\begin{figure}[!t]
\centering
\includegraphics[width=.70\columnwidth]{figures/READEX_workflow.png}
\caption{Overview of READEX methodology}
\label{fig:readex_methodology}
\end{figure}

\subsection{Application Instrumentation and Analysis Preparation}
\label{sec:application_instrumentation}
The first step in the READEX methodology is to instrument the HPC application by inserting probe-functions around different regions that are of interest to tuning. A region can be any arbitrary part of the code, for instance a function or a loop. The READEX tool suite is based on instrumentation with Score-P and requires that the \textit{phase region} be manually annotated. A phase region is single-entry and exit region where most of the application progresses. Typically, a phase region may be the body of the main progress loop of the application. In addition to the phase region, the READEX tool suite supports instrumentation of \textit{user regions} inside the phase region. User regions may be instrumented manually using Score-P or automatically by the compiler.

The READEX methodology also allows exposing parameters that define dynamic behaviour of the application to the READEX tool suite. These parameters, referred to as additional identifiers, will enhance the distinction of different scenarios into runtime situations during the application execution, potentially leading to better identification of optimal configurations for the different tuning parameters. An example of additional identifiers is different input data sets. Exposing this information will allow READEX to detect different compute characteristics of the application caused by different inputs.

\subsection{Application Pre-analysis}
\label{sec:dynamism_detection}
After instrumenting an application and preparing it for analysis, the second step in the READEX approach is to perform a pre-analysis. The objective of this step is to automatically identify and characterise dynamism in the application behaviour. This is critical because the READEX approach is based on tuning hardware, system and application parameters based on the dynamism exhibited by the different regions in the application. The READEX tool suite is capable of identifying and characterising two types of application dynamism:
\begin{itemize}
  \item \textit{Inter-phase dynamism}: This occurs when each phase (execution instance) of a phase region in the application exhibits different characteristics. This results in different values for the measured objective values (execution time) and thus may require different configurations to be applied for the tuning parameters.
  \item \textit{Intra-phase dynamism}: This occurs when each runtime situation (execution instance) of the significant regions in a phase region exhibits different characteristics and results in different values fo the measured objective values (execution time, compute intensity) and thus may need different configurations to be applied for the tuning parameters.
\end{itemize}
The pre-analysis also identifies the regions in the application that contribute significantly to the execution of an application and are referred to as \textit{significant regions}. If no dynamism is identified in the pre-analysis then the rest of the READEX steps should be aborted due to homogeneous behaviour of the application which will not yield any energy or performance savings from auto-tuning.

Listing \ref{lst:minimd_dynamism_summary} presents an example of the summary of significant regions and the dynamism identified by the READEX tool suite in the miniMD application.

\lstset{language=[90]Fortran,
	basicstyle=\ttfamily\scriptsize,
	frame=tb,
	aboveskip=2mm,
	belowskip=2mm,
	showstringspaces=false,
	columns=flexible,
	breaklines=true,
	breakatwhitespace=true,
	keywordstyle=\color{black},
	commentstyle=\color{black},
	escapeinside={(@*}{*@)},
}
\begin{lstlisting}[caption={Summary of Application Pre-analysis},label=lst:minimd_dynamism_summary]
...
Significant regions are:

void Comm::borders(Atom&)
void ForceLJ::compute_halfneigh(Atom&, Neighbor&, int) [with int EVFLAG = 0; int GHOST_NEWTON = 1]
void ForceLJ::compute_halfneigh(Atom&, Neighbor&, int) [with int EVFLAG = 1; int GHOST_NEWTON = 1]
void Neighbor::build(Atom&)


Significant region information
==============================
Region name                  Min(t)          Max(t)       Time Dev.(%Reg) Ops/L3miss    Weight(%Phase)

void Comm::borders(Atom&)    0.001             0.001            2.6           109              0
void ForceLJ::compute_hal    0.013             0.014            2.9            97             68
void ForceLJ::compute_hal    0.016             0.016            0.0            91              1
void Neighbor::build(Atom    0.047             0.048            0.7           332             23


Phase information
=================
Min                  Max                  Mean                 Dev.(% Phase)        Dyn.(% Phase)

0.0138626            0.0664566            0.020337             72.731               258.612

...

SUMMARY:
========

Inter-phase dynamism due to variation of execution time of phases

No intra-phase dynamism due to time variation

Intra-phase dynamism due to variation in compute intensity of following significant regions

void ForceLJ::compute_halfneigh(Atom&, Neighbor&, int) [with int EVFLAG = 0; int GHOST_NEWTON = 1]
void Neighbor::build(Atom&)
\end{lstlisting}

\subsection{Derivation of Tuning Model}
\label{sec:tuning_model_generation}
Following the identification of exploitable dynamism in the pre-analysis step, the third step is to explore the space of possible tuning configurations and identify the optimal configurations of the tuning parameters for the phases and runtime instances during the application execution. This analysis is performed by PTF (Periscope Tuning Framework) in conjuction with Score-P and RRL (READEX Runtime Library). PTF is allows for performing design-time analysis experiments through a number of possible search strategies (eg. exhaustive, individual and heuristic search based on generic algorithm) to identify the optimal configurations for the runtime instances of significant regions identified in the pre-analysis step that exhibit dynamism. To achieve this, Score-P provides the instrumentation and profiling platform, while the RRL provides the platform for libraries to tune hardware and system parameters. Additionally, the READEX tool suite also has dedicated libraries that allow tuning application-specific parameters.

It is important to note that the additional identifiers specified during the instrumentation step are particularly key in providing additional domain knowledge to distinguish and identify different optimal configurations for runtime scenarios \cite{PACO17}.

After all scenarios are explored and optimal configurations are identified, the rts' are grouped into a limited number of scenarios, ef. upto 20. Each scenario is associated with a common system configuration, and it is hence composed of rts' with identical or similar best configurations. The limitation in the number of scenarios inhibits a too frequent configuration switching at runtime that may result in higher overheads from auto-tuning. The set of scenarios, information about the rts' associated with the scenarios, and the optimal configurations for each scenario are stored by PTF in the form of a serialised text file called the Application Tuning Model (ATM), to be loaded and exploited during production runs at runtime.

\subsection{Runtime Application Tuning}
\label{sec:runtime_tuning}
Following the completion of design time analysis, production runs of the application can now be tuned at runtime using the optimal configurations summarised in the ATM using the READEX Runtime Library (RRL). The RRL monitors the application execution using the Score-P instrumentation, identifies the scenario that is encountered at runtime, and applies the corresponding optinmal configurations for each scenario using the knowledge in the ATM to optimise the application's energy consumption. The RRL uses libraries that are loaded as plugins for setting different setting configurations for the parameters that are tuned by the READEX tool suite.

\subsection{Pathway for READEX Workflow}
\label{sec:pathway_for_readex_workflow}
Pathway \cite{Pathway:Petkov13} is a tool for designing and executing performance engineering workflows for HPC applications. In the scope of READEX project, Pathway provides an out-of-the box workflow template that can be configured to apply the READEX tool suite on an application in a HPC system of choice. Fig. \ref{fig:pathway_workflow} illustrates the READEX workflow available in Pathway and Fig. \ref{fig:pathway_browser} presents an example of a custom browser in Pathway that summarises the results from each step of applying the READEX tool suite on the miniMD application.

\begin{figure}[!t]
\centering
\includegraphics[width=.95\columnwidth]{figures/PathwayWorkflow.png}
\caption{READEX Workflow in Pathway}
\label{fig:pathway_workflow}
\end{figure}

\begin{figure}[!t]
\centering
\includegraphics[width=.95\columnwidth]{figures/PathwayBrowser.jpeg}
\caption{READEX Results Browser in Pathway}
\label{fig:pathway_browser}
\end{figure}

\section{Design-Time Analysis} \label{sec:dta}
\textbf{TUM (4 pages)} 

\textcolor{blue}{\begin{enumerate}
  \item Describe how the intra-phase or inter-phase plugin should be selected depending on the dynamism that has to be exploited. 
  \item Write how PTF uses tuning steps to perform experiments
  \item Configuration file for objective to tune and search algorithms to use
\end{enumerate}
}

\subsection{Intra-Phase Plugin} \label{sec:intra-phase}

PTF performs intra-phase dynamism tuning by executing \textbf{readex\_intraphase} plugin when there is no changes of the dynamic characteristics across the sequence of phases.

The plugin can be configured through the \textit{READEX configuration file} where the user can specify significant regions, objectives, system-level tuning parameters and search strategies. The significant regions of an application are produced by \textbf{readex-dyn-detect}~\cite{kumaraswamy2018design} which posses certain dynamism in different characteristics. The objective functions are used to evaluate the effect of different configurations of the tuning parameters. Different supported objective functions are absolute energy consumption, CPU energy consumption, execution time, energy delay product limiting the performance reduction due to energy tuning and total cost of ownership.  The total cost of ownership the summed up cost of the energy in addition to the execution time dependent fraction for hardware and software investment as well as maintenance costs and personnel. 

The normalized form of all the above objectives is the normalized objective value with respect to a metric, for example the number of AVX instructions. These normalized objectives are applied to the plugin that have different regions in different phases but no varying characteristics such as the computational intensity. The normlaized objective would allow to compare the objective values of phase with different number of significant regions. 

The plugin supports system-level tuning parameters such as core and uncore frequency and the number of threads as well as application tuning parameter. The ranges of different tuning parameter can also be specifed to apply for evaluation.

The plugin also allows the user to specify different search algorithms such as exhaustive, individual, random, genetic and so on. The additional parameters for each of the alogrithms can also be specified via this configuration file.

The tuning steps of the \textbf{readex\_intraphase} plugin are as follows:

The first tuning step of PTF is executed with the default configuration of the tuning parameter to collect the program's static information after starting the application. PTF uses a specific analysis strategy to gather program regions and runtime situations only for the first phase of the application. The measurement results are required to evaluate the objective value, for example, time and energy and stored to compare with the results of the last tuning step.

The next step is to tune ATPs as they are specifically independent of the system-level tuning parameter. The application expert can provide such kind of application specific parameters for example algorithmic choices. The detail about ATP is described in section ~\ref{sec:atp}. The plugin provides two new search strategies, \textbf{exhaustive\_atp} and \textbf{individual\_atp} to compute the optimal ATP configuration. The exhausive search space is built from the crossproduct of all valid combinations of ATPs and the idividual stratgey is computed from the domains individually. It first evaluates all valid points for the first domain. The best point from this domain remains fixed and investigates the next domain untill al the domains were explored. The search strategy then used \textbf{exhaustive\_atp} to explore the valid points of a domain. PTF uses these two search strategies to compute the optimal configuration for the given ATPs.   

The third step is the system-level parameter tuning which investigates the optimal configuration for system-level tuning paramter keeping the optimal configuration of the ATPs fixed found during the previous step. The search strategy is read from the configuration file. If no strategy us specified, the default individual search algorithm is selected on this step. The experiments are created for all possible ranges of the system-level tuning parameters. The plugin then evaluates the objective for the phase region followed by for each rts to compute the best system configuration respectively. The READEX tuning model is generated from this knowledge. 

The last step is called as the verification step which is performed by executing additional three experiments in order to check for phase variations with the results produced in the previous step. For this purpose, PTF configures RRL with the best system configuration for phase region and with the rts-specific best configuration and switches configurations dynamically.
 
\subsection{Inter-Phase Plugin} \label{sec:inter-phase}

\section{Tuning Model Generation} \label{tm-generation}
\textbf{NTNU (2 pages)}

\textcolor{blue}{\begin{enumerate}
    \item Creation of scenarios to reduce switching overhead
	\item How does the selector work 
  	\item Cost function / trade-off 
\end{enumerate}
}


\section{Tuning Model Visualization} \label{sec:tm-visualization}
\textbf{TUM (1 page)}
The effect of the best configuration of different tuning parameters can be inspected by comparing different scenarios of the tuning model and explains how similar scenarios appear in closer proximity, while dissimilar scenarios are apart.

The \textit{Forced layout graph} is used to visualize the tuning model result. The graph is constructed based on the JavaScript library D3.js~\cite{bostock2011d3}. It compares scenarios in the tuning model with respect to their similarity and weight. In this context, similarity represents the distance of scenarios in a multi-dimensional tuning space, and weight is the aggregated execution time of \textit{rts's} of a scenario relative to the phase execution time. While similarity is represented by the thickness of the edges between scenarios, the weight is visualized as the size of the circle representing a scenario. Eventually, the distance between scenarios is the result of all forces. The network adapts according to the forces dynamically. 

\begin{figure}
	\begin{mdframed}
	\centering
		\includegraphics[width=0.60\textwidth]{figures/luleshTM_expand.jpg}
	\end{mdframed}
	\caption{\label{fig:forced-layout-expand}The expanded forced layout of the tuning model of LULESH upon clicking on a scenario node }
\end{figure}

Figure~\ref{fig:forced-layout-expand} shows the tuning model of the LULESH proxy application from the CORAL benchmark suite. The nodes in the figure represent scenarios found in tuning model. Each node is a cluster of \textit{rts's} belonging to it. There are six scenarios in LULESH's tuning model where \texttt{Scenario~1} covers most of the execution time. On the other hand, \texttt{Scenario~2} and \texttt{Scenario~4} are the least significant nodes due to their lowest weights. As the figure shows, \texttt{Scenario~1} and \texttt{Scenario~2} are the most similar scenarios and high thickness of the edge and lowest distance between them affirms that. On the opposite side, \texttt{Scenario~3} and \texttt{Scenario~6} are the most distant dissimilar scenarios.

To investigate each scenario, the user can click on a scenario node of the graph. Upon clicking on a node, the node expands with all the rts nodes of that scenario. A pop over box appears upon hovering on the node which shows the scenario information including rts's with their weight and configurations of the tuning parameters. In this figure, \texttt{Scenario~1} contains two rts's: \texttt{rts 1} and \texttt{rts 2} each representing 18.35\% and 17.31\% weight of the phase respectively. The application tuning parameter is yet to be implemented into ATM.
\section{Runtime Application Tuning} \label{rat}

Once the DTA analysis is finished and the ATM has been generated, the application can be used in production. 

The Runtime Application Tuning (RAT) phase of READEX is carried out by the low-overhead READEX Runtime Library (RRL). 

The RRL is implemeted as a Score-P Substrate Plugin using the Substrate Plugin Interface \cite{Schoene2017}. This plugin interface allows to utilize the instrumentation infrastructure of Score-P, without direct integration into Score-P. This approach acts to reduce maintenance and integration efforts by keeping the RRL as a separate entity. As a substrate plugin, the RRL receives notifications for differents events occuring during the application run through various means of instrumentation and uses this information to make switching decisions based on the application tuning model created at design time.

At runtime, the RRL obtains the set of configurations, in the form of scenarios, classifiers and selectors, from the ATM generated by PTF at design-time.

The RRL implements three main mechanisms in order to apply the dynamic configuration switching at runtime, namely, scenario detection, configuration switching and calibration. The following sections detail the first two mechanisms, while Section~\ref{sec:calibration} describes the calibration mechanism, which is an extension to the standard version. Figure~\ref{fig:rrl} shows the detailed architecture of RRL, and will be used to explain the three mechanisms mentioned above. 

%\begin{figure}[!t]
%\centering
%%\includegraphics[trim={7cm 2cm 5.5cm 2cm},clip,width=3in]{readex-approach}
%\includegraphics[width=.8\columnwidth]{figures/RRL_Architecture.png}
%\caption{Architecture of the READEX Runtime Library (RRL)}
%\label{fig:rrl}
%\end{figure}

\subsection{Scenario Detection}\label{scenario-detection}
 Runtime detection of the upcoming scenario is a process that involves several modules in the RRL architecture depicted in Figure~\ref{fig:rrl}. A production run of an application starts with a loading of the ATM into the Tuning Model Manager (TMM). When a new region is entered during the application execution, the Control Center receives notification from Score-P. The Control Center passes this information to the RTS handler, which asks if the region encountered is significant from TMM. If the TMM can find the region in ATM, then it is a significant region otherwise it is marked as an unknown region. The RTS Handler next checks the granularity of the current region. If the granularity of the region is above 100ms, only then the region is tuned. If the region is significant and has granularity above threshold, the RTS Handler first requests for the number of additional identifiers from the TMM. Once the required number of additional identifiers are collected by RTS Handler through \textit{user parameter} event notifications, the current rts is identified by both the call stack (maintained in the RTS Handler) and the additional identifiers.
RTS Handler then requests a new configuration for the current rts from the TMM which is then passed to the Parameter Controller by RTS handler to apply the configuration switching.  
If it is an unknown region and the granularity is above the threshold of 100ms, then the calibration mechanism is invoked.

\subsection{Configuration Switching}\label{config-switching}
 Switching of parameter settings is performed through the parameter control plugins that are also employed during the DTA phase.
During scenario detection, if the region entered is found to be significant by the RTS Handler and has high enough granualrity, the RTS Handler gets a new configuration from the TMM for the current rts. 
This configuration is passed to the Parameter Controller, which sets the configuration through the respective Parameter Control Plugins (PCP).

Before the current region exits, the RTS Handler receives the notification  from Score-P through Control Center. 
The RTS Hander checks if the current region was set up for calibration. If yes, it requests the configuration for the currently exited region from the
Calibration module. Once the RTS handler gets back the configuration from the calibration module, it passes this configuration to the TMM which stores the new configuration for the respective rts. If the region was not set up for calibration, then the Parameter Controller is informed that it might want to unset the current configuration. 

The Parameter Controller supports two different modes: \textit{reset} and \textit{no-reset}. The first mode maintains a configuration stack. Whenever a new configuration is set, the previous configuration is pushed onto this stack. When the corresponding unset occurs, the element is removed from the stack and the previous configuration is set. If the no-reset mode is selected, the current configuration stays active until a new configuration is set. The unset is ignored. This behaviour is configurable by the user. By default, \textit{reset} mode is enabled.

\subsection{Calibration}\label{calibr}
 For the already seen rts's, RRL extracts the optimal configuration from the ATM. For the un-seen ones, the RRL calibration mechanism, explained in Section~\ref{sec:calibration}, is used to find the optimal system configuration based on machine learning algorithms and the data stored in the tuning model. 

 
\section{Dynamic Switching Visualization} \label{switching-visualization}
The configuration switching happens during both phases of the READEX methodology. 
During DTA, PTF runs experiments with different configurations to find the optimal configuration for each rts in the application, which are then stored in the tuning model. During RAT, RRL applies the optimal configuration from the tuning model for each rts during the application run. Both stages require configuration switching during the application run.

To enable the user to visualize the configuration switching for each region during DTA and during a production run, a switching visualization module is included in the RRL. 
The visualization module is implemented as a Score-P Metric Plugin. The metric plugin uses the Metric Plugin Interface provided in Score-P \cite{Schoene2017}. The user can select any of the hardware, software and application tuning parameters to visualize the switching pattern. The tuning parameters selection is configurable and the user can specify if all the tuning parameters or a subset has to be recorded. Each of the tuning parameters is then added as a metric and recorded in a trace in the \textit{OTF2} format \cite{Ilsche-Cstate} by Score-P. The trace can be visualized in the Vampir \cite{BHJR:10:VampirOverview}. 
 
Figure~\ref{fig:switch_visualization} illustrates the switching of the CPU frequency and uncore frequency performed by RRL while tuning CPU frequency and uncore frequency for the Blasbench benchmark. \todo{Explain a bit more}
\begin{figure}[!t]
\centering
%\includegraphics[trim={7cm 2cm 5.5cm 2cm},clip,width=3in]{readex-approach}
\includegraphics[width=.95\columnwidth]{figures/visualization_trace.png}
\caption{{CPU\_FREQUENCY} and {UNCORE\_FREQUENCY} switchings during Blasbench runtime tuning}
\label{fig:switch_visualization}
\end{figure}

 
\section{Extensions} \label{sec:extensions}

READEX provides additional means for tuning the energy efficiency of applications extending the presented concepts. These cover the tuning of application level parameters which select different algorithms, for example, and provide a significant extension of the tuning parameter space. Application-level tuning parameters can either be declared in the program code but also be tuned if they are passed to the program via the input files. READEX also supports the construction of a generic tuning model from tuning models for program runs with different inputs. The last extension presented here covers runtime tuning for situation that were not seen during design-time called calibration. 






\subsection{Application Tuning Parameters} \label{sec:atp}
\textbf{Intel (1 page)}
\textcolor{blue}{\begin{enumerate}
	\item Motivation
	\item How does this work? (A high level workflow, maybe?)
	\item Explain that ATPs are combined to domains and those in a domain can have constraints. 
	\begin{itemize}
		\item Manual insertion into code (ATP library). Can use runtime info to set e.g. the value space of an ATP.
		\item Pre DTA: Generation of ATP file
		\item During DTA: ATP server providing valid settings for ATPs based on the value space and the constraints.
		\item During RAT: Read by ATP library from the RRL
	\end{itemize}
\end{enumerate}}
\subsection{Application Configuration Parameters} \label{sec:acp}

Application level tuning parameters are frequently given in the input files configuring the program run. To simply tuning those Application Configuration Parameters (ACP) READEX provides an additional tuning plugin. 

The \texttt{readex\_configuration} tuning plugin enables tuning of ACPs with respect to one of the objectives supported by READEX. The plugin first reads a plugin specific configuration file. This specifies the objective, the search algorithms, and the tuning parameters. ACPs are identified by their name in the input file. For each such input file a template file with the name is given. During the search for the optimal configuration for the ACPs, the plugin copies the template file to the input file and replaces all ACP names with the value given in the selected configuration. It then restarts the application and measures the resulting objective value for the phase region. Only a single phase is required, but also a burst of phases can be used in the experiment. The READEX User's Guide in the appendix provides details of this procedure. 

The final outcome of this tuning plugin is an optimal configuration for the ACPs which is output into a file. Furthermore, final input files are created from the template files by replacing ACPs by their value in the best configuration.  

\subsection{Input Identifiers} \label{sec:input}

DTA exploits the variations in the applications for different input sets to improve the tuning model by identifying more rts's with different characteristics. Different application characteristics induced by the application inputs can be passed to READEX via input identifiers in the form of key-value pairs in a separate file. For example, in the MG benchmark~\cite{npb}, the maximum grid resolution may change the compute intensity on the different grid levels. The change from memory to compute bound on coarser grids might happen later if the resolution of the finest gird is higher. To be more precise, the combination of the finest resolution and the number of processes onto which the grids are distributed influences the computational characteristics. As the number of processes increase, data distributes better over the caches, resulting in switching between memory bound and compute bound. Hence, the numbers of processes may also be considered as an input identifier. 

Each input identifier is attached to a specific ATM while generating the tuning model at the end of the \texttt{readex\_intraphase} plugin. In order for all of the tuning information from the different ATMs to be usable by the RRL, these tuning models must be merged into a single tuning model. This is performed by the \texttt{tuning model merger}.

The \texttt{tuning model merger} is a standalone program that takes all the ATMs as input on its command line and outputs a new ATM incorporating all the rts's from the input ATMs. The program does this by first deserializing all ATMs. It extracts all tuning information from the ATMs, such as rts's and their system configurations as well as the corresponding input identifiers. The scenarios from the individual ATMs are discarded. Next, the tuning model merger filters all rts's in order to avoid duplicated rts's in the final tuning model. The next step is to produce a new set of scenarios by clustering all rts's as described in Section \ref{tm-generation}. Finally, the \texttt{tuning model merger} serializes the merged tuning model information and outputs the new ATM in the JSON format.

\subsection{Runtime Calibration} \label{sec:calibration}
During RAT, we differentiate between known and unknown rts's. 
Known rts's are those which have been encountered during DTA. So the optimal configuration for the rts's is known. 
However, unknown rts's describe those which have not been encountered during DTA. 
There are several reasons why unseen rts's might occur. 
First, an unseen rts may consists of already known regions, but with unknown user parameters that were not present during DTA or parameters that might have changed between DTA and runtime. Typically, these changes are related to different application inputs. 
Second, an unseen rts may consist of completely new regions, which were not seen during DTA. 
The goal of the calibration is to handle these unknown rts's during RAT.

Since, the calibration is done during the production run, there are a few restrictions which had to be taken in account for the design of calibration mechanism. 
First, there can be no user input and 
second, a good configuration has to be found in short time. 
A DTA like approach for searching optimal configuration is not feasible as it would degrade the performance of the application.
To avoid this a Machine Learning based method is used to
determine a good configuration for an unseen rts. 
Using this method we could split the calibration in a training part and a detecting part. 
The training is done once per HPC system and described below. During RAT the trained model is used to detect a good configuration.
Once a configuration is found, it is stored in the TMM.

\subsubsection{Data Basis:}
Each Machine Learning algorithm needs a data basis, also called \textit{feature vector}, to learn from. 
For supervised learning, an \textit{optimization criterion} and a \textit{target vector} are needed as well. 
In our case, the \textit{optimization criterion} is to reduce the energy consumption of certain program functions. 
To do so, we change the frequency of the processor core and uncore, which represents the \textit{target vector}. 
The training examples are generated by different energy optimal frequencies for monitored program functions. 
As feature vector, we use the hardware performance measurement counters (PMCs) as the data basis to learn from and predict a
good configuration.

PMCs are CPU registers that can be used to count different hardware events, like executed instructions, cache accesses, and branch predictions.
Modern processor architectures are equipped with counters that measure events related to the processing units and core-related caches, and counters that observe the behavior of shared components within the uncore \cite{molka:2017:a}.
A detailed description for core and uncore counters on the Intel
Haswell-EP platform can be found in \cite{Intel2018} and \cite{xeon_e5v3_Uncore}, respectively.

The purpose of the learning algorithm is to use these counters to predict the most energy efficient configuration of the processor frequency for a specific region. However, as there is only a limited number of PMCs available to measure the different events, it is not possible to collect all events at the same time. Hence, we need to find the appropriate core and uncore events.

\subsubsection{Selecting Relevant Performance Events:}

Suitable hardware events are those that contribute to a decision about the optimal frequencies.
However, to gain the most accurate information, we need to filter out redundant information. 
We use the \textit{Correlation-based feature selection} approach proposed by Yoo \cite{Yoo2012} to choose events without reduntant information. 
To generate the input vectors for the learning algorithm, we remeasure
the chosen relevant events to generate new feature vectors.

\subsubsection{The Learning algorithm:}
For the implementation of the learning algorithm we chose Neural Networks (NN). 
Neural networks try to mimic the human brain, where computational units often referred to as \textit{neurons} combine input values to produce an output value \cite{Haykin2009}.
The aim is to collect feature vectors for a given configuration of the processor frequency that results in the lowest energy consumption. Moreover, in the training step, this data is normalized by the runtime of the region and used as input to the NN. The optimal processor
frequency for each region is termed as output.

\subsubsection{Training and Runtime:}
For tuning of the ML algorithm, the PTF is replaced by the tuning framework in Figure~\ref{fig:rrl} and PMCs are given as input. 
For training, SPEC OpenMP benchmark has been used. The benchmarks are run on our target platform \ref{sec:results} and different performance counters are collected. 
At the end of each benchmark run, the collected counter
data is saved and evaluated by the training framework. 
If the collected counter data are sufficient, the framework will proceed. Otherwise, the benchmark is executed again. 
Once the training satisfies the quality requirements set, the framework determines which events are the most interesting ones.

Now, at runtime during a production run, when the RTS-Handler
detects the presence of an unseen rts, instead of using a system configuration found during DTA, the calibration module specifies a configuration and starts collecting values from the pre-selected set of counters.
At a pre-determined point in the program, for example, after 100 iterations, the collected counter values are fed to the learning algorithm, which then determines the best configuration to apply for future occurrences of the same rts.
\section{Results} \label{sec:results}
%%%%%%%%%%%%%%%%%%%%%%%%%%%%%%%%%%%%%%%%%%%%%%%%%%%%%%%%%%%%%%%%%%%%%%%%%%%%%%%%%%%%%%%%%%
%\textbf{IT4I (3 pages)}
%\textcolor{blue}{\begin{enumerate}
%	\item Description of the machine/modules
%	\item Rational behind selecting application classes, i.e., benchmarks, proxy apps, applications
%	\item Brief description of the dynamism in the selected applications
%	\item Add the results that were presented at the review meeting. Focus on the table presenting the overall results and not specifically to a single application. 
%	\item Do not compare to the project goals but present the achievements. e.g. min, max, avg compared to manual tuning. 
%	\begin{itemize}
%		\item A table summarizing the savings - improvements
%		\item Should add inter-phase results to this - TUM
%		\item Describe the overall effect of using the READEX methodology
%	\end{itemize}
%	\item Results for ATPs
%\end{enumerate}}
%%%%%%%%%%%%%%%%%%%%%%%%%%%%%%%%%%%%%%%%%%%%%%%%%%%%%%%%%%%%%%%%%%%%%%%%%%%%%%%%%%%%%%%%%%
To demonstrate that READEX is capable of supporting different architectures and software stacks, we tested it on the Intel Haswell and Intel Broadwell processors both at TU Dresden's Top500 cluster Taurus\footnote{\url{https://doc.zih.tu-dresden.de/hpc-wiki/bin/view/Compendium/SystemTaurus}}. Taurus' Haswell (two Xeon E5-2680 v3 sockets on a single node, with 12 cores each) partition was selected especially due to the reliable power measurement infrastructure called HDEEM~\cite{hdeem} that was used for energy measurement in this project. For energy measurements on the Broadwell (two Xeon E5-2680 v4 sockets, with 14 cores each) partition, we used the Intel RAPL counters with 75\,W baseline~\footnote{The baseline for the Broadwell partition has been established based on low frequency measurements from IPMI.} to estimate the energy consumption of not only the CPUs but the whole node similar to HDEEM.

%Another site where we did the tests is Haswell (two Xeon E5-2680 v3 on single node, 12 cores each) based cluster Salomon operated by IT4Innovations national supercomputing center, in this case the Intel RAPL counters had been used for energy measurement. Taurus as well as Salomon system are listed in the Top500 list most power-full supercomputers.

%%%%%%%%%%%%%%%%%%%%%%%%%%%%%%%%%%%%%%%%%%%%%%%%%%%%%%%%%%%%%%%%%%%%%%%%%%%%%%%%

The following text presents the energy savings that were achieved when using the READEX methodology on READEX test applications, as well as full-fledged applications. The READEX test applications consist of three basic benchmarks: Kripke, Lulesh and Blasbench. BEM4I and ESPRESO are full-fledged applications, whose results are presented in more detail in this section. 

BEM4I~\cite{ch6_MerZap2013} is a solver for Partial Differential Equations (PDEs) based on the Boundary Element Method (BEM), and is under development at IT4Innovations. Contrary to finite element solvers, BEM4I produces dense matrices, and due to the nature of boundary integral equations, the assembly of system matrices is more or less compute bound. In contrast, the iterative solver used for the solution of the resulting system of linear equations is usually memory bound due to matrix vector multiplications.

The ESPRESO library~\cite{ESPRESOijhpca} is a combination of Finite Element (FEM) and BEM tools and domain decomposition solvers. It supports FEM and BEM (uses BEM4I library) discretization for Advection-diffusion equation, Sto\-kes flow and structural mechanics. The ESPRESO solver is a parallel linear solver, which includes a highly efficient MPI communication layer for inter-node communication, and OpenMP for intra-node communication.

\begin{table}[t]
    \centering
%	\resizebox{\textwidth}{!}
%	{%

    \begin{tabular}{|c|c|cH|}
    \hline
            &       Broadwell &         Haswell &         IT4I HSW \\
            &     energy/time &     energy/time &      energy/time \\ \hline
AMG2013	    &   7.5\%/-10.5\% &   7.0\%/-14.0\% &   3.4\%/-23.2\%  \\ \hline
Blasbench   &  12.0\%/-19.0\% &   9.9\%/ -9.2\% &  10.9\%/-19.8\%  \\ \hline
Kripke	    &   4.3\%/-10.3\% &  10.5\%/-28.9\% &  10.3\%/-22.2\%  \\ \hline
Lulesh	    &  10.0\%/ -9.2\% &  18.2\%/-25.7\% &   7.3\%/-20.6\%  \\ \hline
BEM4I	    &  23.0\%/ -1.1\% &  34.0\%/ 10.9\% &  24.3\%/  8.2\%  \\ \hline
INDEED	    &  14.0\%/-18.0\% &  19.1\%/-17.3\% &             -    \\ \hline
NPB3.3-BT-MZ&   8.9\%/-11.3\% &  10.8\%/-12.0\% &   3.1\%/ -2.8\%  \\ \hline
ESPRESO	    &           -     &   7.1\%/-12.3\% &   8.1\%/ -7.0\%  \\ \hline
OpenFOAM    &   7.5\%/ -7.6\% &   9.8\%/ -9.8\% &  18.4\%/ -7.5\%  \\ \hline
    \end{tabular}

%	}
    \caption{Overall energy and time savings achieved using the READEX methodology on the applications for the Broadwell and Haswell platforms.}
    \label{tab:overall2}
\end{table}
Table~\ref{tab:overall2} shows how the runtime and energy consumption changed, when READEX was used to tune the selected applications. Achieved energy savings vary between 4.3\,\% and 34\,\%. 
The BEM4I library showed the best energy savings from the evaluated applications, and in case of the evaluation on the Haswell nodes, the tuned runs were also shorter than the runs without tuning.


\subsection{Exploitation of Application Dynamism}
Since BEM4I resulted in the best energy savings, this section describes in detail how READEX was used to exploit the application dynamism.

PDEs are often used to describe phenomena such as sheet metal forming, fluid flow, and climate modeling. One of the numerical approaches to solving PDEs is BEM implemented in the BEM4I library. In contrast to volume based methods, such as the finite element/differences/volume methods, BEM gives dense matrices whose assembly results in a compute bound code. This fact is even more pronounced when the assembly kernels are parallelized and vectorized as in the case of BEM4I~\cite{ch6_ZapMerMal2016,ch6_MerZapJar2016}. In contrast, the iterative GMRES solver based on the matrix-vector product as implemented in the Intel Math Kernel Library (MKL) is much less compute intensive and results in memory bound computation. Furthermore, printing the results for visualization leads to an I/O bound region. 

For the memory bound solver (GMRES), manual tuning resulted in a low core frequency, high uncore frequency and the use of eight threads to overcome Non-Uniform Memory Access (NUMA) effects of the dual socket computational node. 

%\begin{table}[h]
%    \centering
%    \begin{tabular}{|c|c|c|c|c|c|}
%    \hline
%Compute node energy &	assemble\_k & assemble\_v & gmres\_solve & print\_vtu & main  \\ \hline
%default energy  	&	1467\,J &	1484\,J &	2733\,J &	1142\,J &	6872\,J \\ \hline
%static tuning energy&	1962\,J &	2015\,J &	1366\,J &	 420\,J &	5792\,J \\ \hline
%dynamic tuning energy&	1476\,J &	1462\,J &	1259\,J &	 293\,J &	4531\,J \\ \hline
% \hline
%\textbf{static savings}  	&	-33.8\%	& -35.8\% & 50.0\% & 63.2\% & 15.7\% \\ \hline
%\textbf{dynamic savings} 	&	-0.6\%	&   1.5\% & 53.9\% & 74.3\% & 34.0\% \\ \hline
%    \end{tabular}
%    \caption{Comparison of the BEM4I regions energy consumption in the application default, optimal static and dynamic configurations.}
%    \label{tab:BEM4Idynamicity}
%\end{table}

\begin{table}[h]
    \centering
	\resizebox{\textwidth}{!}
	{%
		\begin{tabular}{|c|c|c|c|c|c|}
		\hline
			 &	assemble\_k & assemble\_v & gmres\_solve & print\_vtu & main \\ 
			 & [J]/[s]     & [J]/[s]    & [J]/[s]     & [J]/[s]   & [J]/[s] \\ \hline
		default setings	&	1467/5.4 &	1484/ 5.9 &	2733/10.2 &	1142/5.6 &	6872/27.3 \\ \hline
		static tuning	&	1962/9.8 &	2015/10.6 &	1366/ 6.1 &	 420/2.4 &	5792/29.0 \\ \hline
		dynamic tuning	&	1476/7.0 &	1462/ 7.2 &	1259/ 7.9 &	 293/2.1 &	4531/24.3 \\ \hline
		 \hline
		static savings [\%]  & -33.8/-82.3	& -35.8/-79.1 & 50.0/40.5 & 63.2/56.8 & 15.7/-6.2 \\ \hline
		dynamic savings [\%]	&  -0.6/-30.6	&   1.5/-20.9 & 53.9/23.2 & 74.3/62.9 & 34.0/10.9 \\ \hline
		\end{tabular}
	}
    \caption{Comparison of the energy and time consumption for the default, optimal static, and dynamic configurations of BEM4I.}
    \label{tab:BEM4Idynamicity2}
\end{table}
The energy and time consumption of each region in the application in optimum static and optimum dynamic (optimum configuration for the region itself) configuration is presented in Table~\ref{tab:BEM4Idynamicity2}. While static savings reached 15.7~\%, the dynamic switching among individual configurations increased the savings to 34.0~\% on the Haswell nodes. Decrease in the run time in this case was caused by NUMA effects of the MKL solver -- the tuned version runs on eight threads, and due to the compact affinity, all threads run on a single socket. It can also be seen that the optimum static configuration has very bad impact on the \texttt{assemble\_k} and \texttt{assemble\_v} regions, and also resulting in a suboptimal behavior of the region \texttt{print\_vtu}.


\subsection{Application Parameters tuning}
As mentioned in Section~\ref{sec:extensions}, READEX comes with two approaches to tune application parameters: (1) using the ATP library, and (2) using Application Configuration Parameters (ACP). The integration of the ATP library requires developer knowledge of the application and therefore, we implemented this support into the ESPRESO library, which was developed by IT4Innovations in the READEX project.

A long list of ATPs were evaluated: runtime tuning of FETI METHOD (2 options), PRECONDITIONERS (5 options), ITERATIVE SOLVERS (2 options), HFETI type (2 options), SCALING (2 options), BO\_TYPE (2 options), NON-UNIFORM PARTS (6 options), REDUNDANT LAGRANGE (2 options) and adaptive precision (2 options). For runtime tuning of domain decomposition (10 options), the developer had to implement the support for this parameter, since ESPRESO performs domain decomposition only during startup. For READEX, we developed an enhanced ESPRESO to redo the decomposition after each time-step of a transient simulation. The resulting total number of possible combinations was 3840.

Besides the ESPRESO library, we analyzed three other applications using the ATPs or ACPs. The energy savings are presented in Table~\ref{tab:ATPACP}.

\begin{table}[h]
    \centering

    \begin{tabular}{|c|c|c|c|}
    \hline
Application & \# parameters tested /& Energy savings     & Energy savings \\
            & total \# of options      & vs. worst settings & vs. default* settings\\
\hline
ESPRESO     & 9/3840 & 86\% & 50--66\% \\ \hline
ELMER       & 1/40   & 97\% & 50--75\% \\ \hline
OpenFOAM    & 2/12   & 24\% &      8\% \\ \hline
INDEED      & 3/12   & 35\% &     25\% \\ \hline

    \end{tabular}
    \caption{Energy savings achieved for the optimal settings over the worst and the default settings after evaluating the  applications with READEX ATPs/ACPs. *In cases where the default settings were not available, the values refer to any reasonable settings.}
    \label{tab:ATPACP}
\end{table}

Since there is no default configuration in ESPRESO, the user has to define the FETI solver based on the knowledge of the problem that has to be solved. Hence, the savings against the default configuration are not presented. Instead, the energy consumption in the best and the worst case are compared. The worst case scenario took 1320 seconds and consumed about 230 kJ, while the best case scenario consumed 32.5 kJ in 189 seconds. Comparing these two cases gives us 86\% energy savings. If the user specifies some reasonable settings, the energy consumption might be about 50-66\% higher than in the best possible settings.




\section{Conclusions} \label{sec:conclusions}

The READEX project developed a tool suite for dynamic tuning of the energy-efficiency of HPC applications. During design-time, it pre-computes a tuning model that is then used during runtime for switching system configurations when certain regions start. This dynamic approach allows to specifically tune the configuration for individual runtime situations and thus exploits the variation in the application characteristics during execution for energy reduction. 

READEX is based on established tools, i.e., Score-P for instrumentation, monitoring, and runtime tuning actions, and PTF for design-time analysis. In contrast to other tools, DTA is carried out in a single application run since it evaluates potential candidate configurations in single phases. As part of READEX, a novel plugin interface for Score-P was implemented that allows the addition of new functionality to the monitor in a transparent way. The RRL is the first demonstrator of this powerful extension mechanism.

The paper also outlined the results obtained from the tuning system, runtime, and the application parameters for a wide range of benchmarks, proxy applications and real applications. The results of dynamic tuning are clearly application dependent but demonstrate the significant potential of the READEX methodology.

% Acknowledgments always without numbering
\section*{Acknowledgment}
This work was supported by the European Union's Horizon 2020 program in the READEX project (grant agreement number 671657).


% ---- Bibliography ----
%
% BibTeX users should specify bibliography style 'splncs04'.
% References will then be sorted and formatted in the correct style.
 \bibliographystyle{splncs04}
 \bibliography{Tools_Workshop2018}
\end{document}
