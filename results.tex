\section{Results} \label{sec:results}
%%%%%%%%%%%%%%%%%%%%%%%%%%%%%%%%%%%%%%%%%%%%%%%%%%%%%%%%%%%%%%%%%%%%%%%%%%%%%%%%%%%%%%%%%%
%\textbf{IT4I (3 pages)}
%\textcolor{blue}{\begin{enumerate}
%	\item Description of the machine/modules
%	\item Rational behind selecting application classes, i.e., benchmarks, proxy apps, applications
%	\item Brief description of the dynamism in the selected applications
%	\item Add the results that were presented at the review meeting. Focus on the table presenting the overall results and not specifically to a single application. 
%	\item Do not compare to the project goals but present the achievements. e.g. min, max, avg compared to manual tuning. 
%	\begin{itemize}
%		\item A table summarizing the savings - improvements
%		\item Should add inter-phase results to this - TUM
%		\item Describe the overall effect of using the READEX methodology
%	\end{itemize}
%	\item Results for ATPs
%\end{enumerate}}
%%%%%%%%%%%%%%%%%%%%%%%%%%%%%%%%%%%%%%%%%%%%%%%%%%%%%%%%%%%%%%%%%%%%%%%%%%%%%%%%%%%%%%%%%%
To demonstrate that READEX is capable of supporting different architectures and software stacks, we tested it on the Intel Haswell and Intel Broadwell processors both at TU Dresden's Top500 cluster Taurus\footnote{\url{https://doc.zih.tu-dresden.de/hpc-wiki/bin/view/Compendium/SystemTaurus}}. Taurus' Haswell (two Xeon E5-2680 v3 sockets on a single node, with 12 cores each) partition was selected especially due to the reliable power measurement infrastructure called HDEEM~\cite{hdeem} that was used for energy measurement in this project. For energy measurements on the Broadwell (two Xeon E5-2680 v4 sockets, with 14 cores each) partition, we used the Intel RAPL counters with 75\,W baseline~\footnote{The baseline for the Broadwellpartition has been established based on low frequency measurements from IPMI.} to estimate the energy consumption of not only the CPUs but the whole node similar to HDEEM.

%Another site where we did the tests is Haswell (two Xeon E5-2680 v3 on single node, 12 cores each) based cluster Salomon operated by IT4Innovations national supercomputing center, in this case the Intel RAPL counters had been used for energy measurement. Taurus as well as Salomon system are listed in the Top500 list most power-full supercomputers.

%%%%%%%%%%%%%%%%%%%%%%%%%%%%%%%%%%%%%%%%%%%%%%%%%%%%%%%%%%%%%%%%%%%%%%%%%%%%%%%%

The following text presents the energy savings that were achieved when using the READEX methodology on READEX test applications, as well as full-fledged applications. The READEX test applications consist of three basic benchmarks: Kripke, Lulesh and Blasbench. BEM4I and ESPRESO are full-fledged applications, whose results are presented in more detail in this section. 

BEM4I~\cite{ch6_MerZap2013} is a solver for partial differential equations based on the boundary element method, and is under development at IT4Innovations. Contrary to finite element solvers, BEM4I produces dense matrices, and due to the nature of boundary integral equations, the assembly of system matrices is more or less compute bound. In contrast, the iterative solver used for the solution of the resulting system of linear equations is usually memory bound due to matrix vector multiplications.

The ESPRESO library~\cite{ESPRESOijhpca} is a combination of Finite Element (FEM) and Boundary Element (BEM) tools and TFETI/HTFETI solvers. It supports FEM and BEM (uses BEM4I library) discretization for Advection-diffusion equation, Sto\-kes flow and structural mechanics. The ESPRESO solver is a parallel linear solver, which includes a highly efficient MPI communication layer for inter-node communication, and OpenMP for intra-node communication.

\begin{table}[t]
    \centering
%	\resizebox{\textwidth}{!}
%	{%

    \begin{tabular}{|c|c|cH|}
    \hline
            &       Broadwell &         Haswell &         IT4I HSW \\
            &     energy/time &     energy/time &      energy/time \\ \hline
AMG2013	    &   7.5\%/-10.5\% &   7.0\%/-14.0\% &   3.4\%/-23.2\%  \\ \hline
Blasbench   &  12.0\%/-19.0\% &   9.9\%/ -9.2\% &  10.9\%/-19.8\%  \\ \hline
Kripke	    &   4.3\%/-10.3\% &  10.5\%/-28.9\% &  10.3\%/-22.2\%  \\ \hline
Lulesh	    &  10.0\%/ -9.2\% &  18.2\%/-25.7\% &   7.3\%/-20.6\%  \\ \hline
BEM4I	    &  23.0\%/ -1.1\% &  34.0\%/ 10.9\% &  24.3\%/  8.2\%  \\ \hline
INDEED	    &  14.0\%/-18.0\% &  19.1\%/-17.3\% &             -    \\ \hline
NPB3.3-BT-MZ&   8.9\%/-11.3\% &  10.8\%/-12.0\% &   3.1\%/ -2.8\%  \\ \hline
ESPRESO	    &           -     &   7.1\%/-12.3\% &   8.1\%/ -7.0\%  \\ \hline
OpenFOAM    &   7.5\%/ -7.6\% &   9.8\%/ -9.8\% &  18.4\%/ -7.5\%  \\ \hline
    \end{tabular}

%	}
    \caption{Overall energy and time savings achieved using the READEX methodology on the applications for the Broadwell and Haswell platforms.}
    \label{tab:overall2}
\end{table}
Table~\ref{tab:overall2} shows how the runtime and energy consumption changed, when READEX was used to tune the selected applications. Achieved energy savings vary between 4.3\,\% and 34\,\%. 
The BEM4I library showed the best energy savings from the evaluated applications, and in case of the evaluation on the Haswell nodes, the tuned runs were also shorter than the runs without tuning.


\subsection{Exploitation of Application Dynamism}
Since BEM4I resulted in the best energy savings, this section describes in detail how READEX was used to exploit the application dynamism.

Partial Differential Equations (PDEs) are often used to describe phenomena such as sheet metal forming, fluid flow, and climate modeling. One of the numerical approaches to solving PDEs is the boundary element method (BEM) implemented in the BEM4I library. In contrast to volume based methods, such as the finite element/differences/volume methods, BEM gives dense matrices whose assembly results in a compute bound code. This fact is even more pronounced when the assembly kernels are parallelized and vectorized as in the case of BEM4I~\cite{ch6_ZapMerMal2016,ch6_MerZapJar2016}. In contrast, the iterative GMRES solver based on the matrix-vector product as implemented in the Intel Math Kernel Library (MKL) is much less compute intensive and results in memory bound computation. Furthermore, printing the results for visualization leads to an I/O bound region. 

For the memory bound solver (GMRES), manual tuning resulted in a low core frequency, high uncore frequency and the use of eight threads to overcome Non-Uniform Memory Access (NUMA) effects of the dual socket computational node. 

%\begin{table}[h]
%    \centering
%    \begin{tabular}{|c|c|c|c|c|c|}
%    \hline
%Compute node energy &	assemble\_k & assemble\_v & gmres\_solve & print\_vtu & main  \\ \hline
%default energy  	&	1467\,J &	1484\,J &	2733\,J &	1142\,J &	6872\,J \\ \hline
%static tuning energy&	1962\,J &	2015\,J &	1366\,J &	 420\,J &	5792\,J \\ \hline
%dynamic tuning energy&	1476\,J &	1462\,J &	1259\,J &	 293\,J &	4531\,J \\ \hline
% \hline
%\textbf{static savings}  	&	-33.8\%	& -35.8\% & 50.0\% & 63.2\% & 15.7\% \\ \hline
%\textbf{dynamic savings} 	&	-0.6\%	&   1.5\% & 53.9\% & 74.3\% & 34.0\% \\ \hline
%    \end{tabular}
%    \caption{Comparison of the BEM4I regions energy consumption in the application default, optimal static and dynamic configurations.}
%    \label{tab:BEM4Idynamicity}
%\end{table}

\begin{table}[h]
    \centering
	\resizebox{\textwidth}{!}
	{%
		\begin{tabular}{|c|c|c|c|c|c|}
		\hline
			 &	assemble\_k & assemble\_v & gmres\_solve & print\_vtu & main \\ 
			 & [J]/[s]     & [J]/[s]    & [J]/[s]     & [J]/[s]   & [J]/[s] \\ \hline
		default setings	&	1467/5.4 &	1484/ 5.9 &	2733/10.2 &	1142/5.6 &	6872/27.3 \\ \hline
		static tuning	&	1962/9.8 &	2015/10.6 &	1366/ 6.1 &	 420/2.4 &	5792/29.0 \\ \hline
		dynamic tuning	&	1476/7.0 &	1462/ 7.2 &	1259/ 7.9 &	 293/2.1 &	4531/24.3 \\ \hline
		 \hline
		static savings [\%]  & -33.8/-82.3	& -35.8/-79.1 & 50.0/40.5 & 63.2/56.8 & 15.7/-6.2 \\ \hline
		dynamic savings [\%]	&  -0.6/-30.6	&   1.5/-20.9 & 53.9/23.2 & 74.3/62.9 & 34.0/10.9 \\ \hline
		\end{tabular}
	}
    \caption{Comparison of the energy and time consumption for the default, optimal static, and dynamic configurations of BEM4I.}
    \label{tab:BEM4Idynamicity2}
\end{table}
The energy and time consumption of each region in the application in optimum static and optimum dynamic (optimum configuration for the region itself) configuration is presented in Table~\ref{tab:BEM4Idynamicity2}. While static savings reached 15.7~\%, the dynamic switching among individual configurations increased the savings to 34.0~\% on the Haswell nodes. Decrease in the run time in this case was caused by NUMA effects of the MKL solver -- the tuned version runs on eight threads, and due to the compact affinity, all threads run on a single socket. It can also be seen that the optimum static configuration has very bad impact on the \texttt{assemble\_k} and \texttt{assemble\_v} regions, and also resulting in a suboptimal behavior of the region \texttt{print\_vtu}.


\subsection{Application Parameters tuning}
As mentioned in Section~\ref{sec:extensions}, READEX comes with two approaches to tune application parameters: (1) using the ATP library, and (2) using Application Configuration Parameters (ACP). The integration of the ATP library requires developer knowledge of the application and therefore, we implemented this support into the ESPRESO library, which was developed by IT4Innovations in the READEX project.

A long list of ATPs were evaluated: runtime tuning of FETI METHOD (2 options), PRECONDITIONERS (5 options), ITERATIVE SOLVERS (2 options), HFETI type (2 options), SCALING (2 options), BO\_TYPE (2 options), NON-UNIFORM PARTS (6 options), REDUNDANT LAGRANGE (2 options) and adaptive precision (2 options). For runtime tuning of domain decomposition (10 options), the developer had to implement the support for this parameter, since ESPRESO performs domain decomposition only during startup. For READEX, we developed an enhanced ESPRESO to redo the decomposition after each time-step of a transient simulation. The resulting total number of possible combinations was 3840.

Besides the ESPRESO library, we analyzed three other applications using the ATPs or ACPs. The energy savings are presented in Table~\ref{tab:ATPACP}.

\begin{table}[h]
    \centering

    \begin{tabular}{|c|c|c|c|}
    \hline
Application & \# parameters tested /& Energy savings     & Energy savings \\
            & total \# of options      & vs. worst settings & vs. default* settings\\
\hline
ESPRESO     & 9/3840 & 86\% & 50--66\% \\ \hline
ELMER       & 1/40   & 97\% & 50--75\% \\ \hline
OpenFOAM    & 2/12   & 24\% &      8\% \\ \hline
INDEED      & 3/12   & 35\% &     25\% \\ \hline

    \end{tabular}
    \caption{Energy savings achieved for the optimal settings over the worst and the default settings after evaluating the  applications with READEX ATPs/ACPs. *In cases where the default settings were not available, the values refer to any reasonable settings.}
    \label{tab:ATPACP}
\end{table}

Since there is no default configuration in ESPRESO, the user has to define the FETI solver based on the knowledge of the problem that has to be solved. Hence, the savings against the default configuration are not presented. Instead, the energy consumption in the best and the worst case are compared. The worst case scenario took 1320 seconds and consumed about 230 kJ, while the best case scenario consumed 32.5 kJ in 189 seconds. Comparing these two cases gives us 86\% energy savings. If the user specifies some reasonable settings, the energy consumption might be about 50-66\% higher than in the best possible settings.
