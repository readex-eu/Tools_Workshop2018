\section{State of the art} \label{sec:related-work}
\textbf{TUM, 1 page} 

\textcolor{blue}{
\begin{enumerate}
  \item Background and related work.
  \item Tuning DVFS well explored. 
  \item In general difficult due to permissions to read energy measurements and set frequencies, different interfaces, and lack of measurement hardware
  \item Tools available in practice? Energy aware scheduling at LRZ. Anything else?
  \item READEX based on ScoreP and Periscope, interfacing msrsave, libadapt, likwid
  \item READEX provides dynamic tuning in addition to static tuning
\end{enumerate}
}

As energy-efficiency and consumption have now become one of the biggest challenges in HPC, research in this direction has gained momentum. Currently, there are many approaches that employ DVFS in
HPC to tune different objectives. Rojek et al.~\cite{Rojek} used DVFS to reduce the energy consumption for stencil computations whose execution time is predefined. The algorithm first collects the execution time and energy for a subset of the processor frequencies, and dynamically models the execution time and the average energy consumption as functions of the operational frequency. It then adjusts the frequency so that the predefined execution time is respected.

Imes et al.~\cite{Imes} developed machine learning classifiers to tune socket resource allocation, HyperThreads and DVFS to minimize energy. The classifiers predict different system settings using the measurements obtained by polling performance counters during the application run. This approach cannot be used for dynamic applications because of the overhead from the classifier in predicting new settings due to rapid fluctuations in the performance counters. READEX, however, focuses on autotuning dynamic applications.
	
The ANTAREX project~\cite{silvano2016antarex} specifies adaptivity strategies, parallelization and mapping of the application at runtime by using a Domain Specific Language (DSL) approach. During design-time, control code is added to the application to provide runtime monitoring strategies. At runtime, the framework performs autotuning by configuring software knobs for application regions using the runtime information of the application execution. This approach is specialized for ARM-based multi-cores and accelerators, while READEX targets all HPC systems.

Intel's open-source Global Extensible Open Power Manager (GEOPM)~\cite{geopm} runtime framework provides a plugin based energy management system for power-constrained systems. GEOPM supports both offline and online analysis by sampling performance counters, identifying the nodes on the critical path, estimating the power consumption and then adjusting the power budget among these nodes. GEOPM adjusts individual power caps for the nodes instead of a uniform power capping by allocating a larger portion of the job power budget to the critical path. 
	
\textit{Conductor}~\cite{Marathe}, a runtime system used at the Lawrence Livermore National Laboratory also performs adaptive power balancing for power capped systems. It first performs a parallel exploration of the configuration space by setting a different thread concurrency level and DVFS configuration on each MPI process, and statically selects the optimal configuration for each task. It then performs adaptive power balancing to distribute more power to the critical path.

The AutoTune project~\cite{guillen2016dvfs,AutoTune:Book2015} developed a DVFS tuning plugin to autotune energy consumption, total cost of ownership, energy delay product and power capping. The tuning is performed using a model that predicts the energy and power consumption as well as the execution time at different CPU frequencies. It uses the enopt library to vary the frequency for different application regions. While this is a static approach, READEX implements dynamic tuning for rts's.

