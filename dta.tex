\section{Design-Time Analysis} \label{sec:dta}
The output of the pre-analysis step~\cite{kumaraswamy2018design} is stored in a configuration file in the \textit{xml} format. The configuration file consists of tags through which the user can provide specifications for:
\begin{description}
	\item [Tuning parameters:] Specified via the ranges (minimum, maximum, step size, and default) for the CPU frequency, uncore frequency, and the number of OpenMP threads.
	\item [Objectives:] These can be the energy, execution time, CPU energy, Energy Delay Product (EDP), Energy Delay Product Squared, CPU energy, Total Cost of Ownership (TCO), as well as their normalized versions. The normalized versions compute the energy consumption per instruction, and can be used for applications with varying amounts of computation in a phase but no change in the phase characteristics.
	\item [Energy metrics:] These include the energy plugin name and the associated metric names. 
	\item [Search algorithm:] This can be the exhaustive, random or individual search strategy.
\end{description}

DTA is performed by the Periscope Tuning Framework (PTF), which is a distributed framework consisting of the frontend, the tuning plugins, the experiment execution engine and a hierarchy of analysis agents. During DTA, PTF reads the configuration file, and calls a tuning plugin~\cite{AutoTune:Book2015}, which searches the multi-dimensional space of system configurations, each of which is a tuning parameter. The tuning plugin performs one or more tuning steps, in which a user-specified search algorithm determines the set of system configurations that are evaluated. Each tuning step executes experiments to measure the effect of the system configuration on the objective. At the end of each tuning step, the plugin checks if the application should be restarted. After all the tuning steps are completed, the plugin generates tuning advice for the application.

Two new plugins, \texttt{readex\_intraphase} and the \texttt{readex\_interphase} were developed for PTF to exploit the dynamism detected by \texttt{readex-dyn-detect}. If \texttt{readex-dyn-detect} reports inter-phase dynamism for the application, the user is advised to select the \texttt{readex\_interphase} tuning plugin. Both plugins perform Dynamic Voltage and Frequency Scaling (DVFS). However, they use different approaches for DTA, and hence, it is recommended to apply the \texttt{readex\_intraphase} tuning plugin if there is no inter-phase dynamism. Sections~\ref{sec:intra-phase} and~\ref{sec:inter-phase} describe in details the steps performed by the \texttt{readex\_intraphase} and \texttt{readex\_interphase} plugins to exploit the application dynamism.


\subsection{Intra-Phase Plugin} \label{sec:intra-phase}

PTF performs intra-phase dynamism tuning by executing the \texttt{readex\_intraphase} plugin when there are no changes in the dynamic characteristics across the sequence of phases.  

The \texttt{readex\_intraphase} plugin executes multiple tuning steps. First, default execution of the application is performed for the default parameter settings. Next, optimal configurations of the system level tuning parameters for different rts's are determined. Finally, the plugin checks the variations in the results of the previous step and computes the energy savings. The \texttt{readex\_intraphase} plugin performs an additional tuning step if application-level tuning parameters are specified for tuning, as described in Section~\ref{sec:atp}.

\subsubsection{Default Execution} \label{intra-default-execution} 

During this step, PTF executes the plugin with the default configuration of the tuning parameters to collect the program's static information after starting the application. The default settings are provided by the batch system for the system parameters. PTF uses a specific analysis strategy to gather program regions and rts's only for the first phase of the application. The measurement results are required to evaluate the objective value, for example, time and energy, and are used later to compare the results in the verification step.

\subsubsection{System Parameter Tuning} \label{sys-tuning} 

The system-level parameter tuning step investigates the optimal configuration for system-level tuning paramters. The plugin uses a user-defined search strategy to generate experiments. The search strategy is read from the configuration file. Exhaustive search generates a tuning space with the cross-product of the tuning parameters, and leads to the biggest number of configurations that are tested in subsequent program phases. The individual strategy reduces the number significantly by finding the optimum setting for the first tuning parameter, and then for the second, and so on. With the random strategy, the number of experiments can be specified. If no strategy is specified, the default individual search algorithm is selected. The experiments are created for the ranges of the tuning parameters as defined in the configuration file. The plugin evaluates the objective for the phase region as well as the rts's. The best system configuration is determined as the one that generates the least energy consumption. This knowledge is encapsulated in the ATM. 

\subsubsection{Verification} \label{intra-verification} 

The verification step is performed by executing additional three experiments in order to check for variations in the results produced in the previous step. For this purpose, PTF configures RRL with the best system configuration for the phase region and the rts-specific best configuration for the rts's. The RRL switches the system configurations dynamically for the rts's.

The plugin then determines the static and dynamic savings for the rts's and static savings for the whole phase before generating the ATM. The following three values characterize the savings at the end of the execution:

\begin{enumerate}
	\item Static savings for the rts's: The total improvement in the objective value with the static best configuration of the rts's over the default configuration. 
	\item Dynamic savings for the rts's: The total improvement in the objective value for the rts's with their specific best configuration over the static best configuration.
	\item Static savings for the whole phase: The total improvement in the objective value for the best static configuration of the phase over the default configuration.
\end{enumerate}


\subsection{Inter-Phase Plugin} \label{sec:inter-phase}

The \texttt{readex\_interphase} plugin~\cite{PDPTA_18_Kumaraswamy} is used for tuning applications that exhibit inter-phase dynamism, where the execution characteristics change across the sequence of phases. While the \texttt{readex\_intraphase} plugin does not consider similarities in the behavior between different phases, the \texttt{readex\_interphase} tuning plugin first groups similarly behaving phases into clusters, and then determines the best configuration for each cluster. It also determines the best configurations for the rts's of each created cluster.

The \texttt{readex\_interphase} plugin performs three tuning steps: cluster analysis, default execution and verification to first cluster the phases, then execute experiments for the default setting of the tuning parameters, and finally, verify if the theoretical savings match the actual savings incurred after switching the configurations. Sections~\ref{cluster-analysis},~\ref{default-execution} and~\ref{verification} describe the tuning steps in detail.

\subsubsection{Cluster Analysis} \label{cluster-analysis} 

The plugin first reads the significant regions, the ranges of the tuning parameters and the objectives from the READEX configuration file. It then uses the random strategy to create a user-specified number of experiments. In each experiment, the plugin measures the effect of executing a phase with a random system configuration from a uniform distribution~\cite{AutoTune:Book2015} on the objective value. The plugin also requests for PAPI hardware metrics, such as the number of AVX instructions, L3 cache misses, and the number of conditional branch instructions, which are used to derive the features for clustering.

Features for clustering are selected carefully, as they have a high impact in selecting the cluster-best configuration. Since the dynamism in many applications results from the variation in the compute intensity and the number of conditional branch instructions, they were chosen as the features for clustering. The compute intensity is defined by $\frac{\#AVX Instructions}{\#L3 Cache Misses}$~\cite{PDPTA_18_Kumaraswamy}. The plugin first normalizes the features and the objective values for all the phases and the rts's by a metric which is representative of the work done, such as the number of AVX instructions. It then transforms the numeric range of the features to [0,1] scale using min-max normalization.

The plugin then uses the DBSCAN (Density-Based Spatial Clustering of Applications with Noise) algorithm~\cite{Ester-1996} to group points that are closely packed together into clusters, and mark points that lie in low-density regions and have no nearby neighbors as noise. DBSCAN requires the \textit{minPts} and the \textit{eps} parameters to cluster the phases. \textit{minPts} is the minimum number of points that must lie within a neighborhood to belong in a cluster, and is set to 4~\cite{Sander-1998}. \textit{eps} is the maximum distance between any two data points in the same neighborhood, and is automatically determined using the elbow method~\cite{gaonkar2013autoepsdbscan}. The elbow is a sharp change in the average 3-NN Euclidean distances plot, and represents the point with the maximum distance to the line formed by the points with the minimum and maximum average 3-NN distances.

The plugin then selects the best configuration for each cluster based on the normalized objective value. The cluster-best configuration represents one best configuration for all the phases of a particular cluster, and individual best configurations for the rts's in the cluster.

\subsubsection{Default Execution} \label{default-execution} 
The tuning plugin restarts the application and performs the same number of experiments as the previous step. In each experiment, the phase is executed with the default system configuration, i.e., the default settings provided by the batch system for the system tuning parameters. The plugin uses the objective values obtained for the phases and the rts's to compute the savings at the end of the tuning plugin.

\subsubsection{Verification} \label{verification} 
The plugin restarts the application and executes the same number of experiments as the previous tuning steps. In each experiment, the plugin executes the phase with the
corresponding cluster-best configuration and the rts's with their rts-specific
best configurations. Phases that were noise points in the clustering step are executed with the default configuration.

After executing all the experiments, the plugin modifies the Calling Context Graph (CCG)~\footnote{A context sensitive version of the call graph} by creating a new node for each cluster, and then clones the children of the phase region. It stores the tuning results and the ranges of the features for the cluster in each newly created node. 

Like the \texttt{readex\_intraphase} plugin, the \texttt{readex\_interphase} plugin computes the savings as described in Section~\ref{intra-verification}. However, the \texttt{readex\_interphase} plugin aggregates the improvements across all the clusters.

%The tuning model contains the list of clusters generated by the clustering algorithm, the set of phases belonging to each cluster, the ranges of the features that were used for clustering, the list of rts's that were tuned by the plugin, the scenarios into which they are classified, and the best configuration for each scenario.

