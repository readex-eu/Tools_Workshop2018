\section{Runtime Application Tuning} \label{rat}

Once the DTA analysis is finished and the ATM has been generated, the application can be used in production. The Runtime Application Tuning (RAT) phase of READEX is carried out by the low-overhead READEX Runtime Library (RRL). RRL is implemented as a Score-P Substrate Plugin using the Substrate Plugin Interface \cite{Schoene2017}. This plugin interface allows to utilize the instrumentation infrastructure of Score-P, without direct integration into Score-P. This approach acts to reduce maintenance and integration efforts by keeping the RRL as a separate entity. As a substrate plugin, the RRL receives notifications for different events occurring during the application run through various means of instrumentation, and uses this information to make switching decisions based on the application tuning model created at design-time.

At runtime, the RRL obtains the set of configurations, in the form of scenarios, classifiers and selectors, from the ATM generated by PTF at design-time.

The RRL implements three main mechanisms in order to apply the dynamic configuration switching at runtime, namely, scenario detection, configuration switching and calibration. The following sections detail the first two mechanisms, while Section~\ref{sec:calibration} describes the calibration mechanism, which is an extension to the standard version.

%\begin{figure}[!t]
%\centering
%%\includegraphics[trim={7cm 2cm 5.5cm 2cm},clip,width=3in]{readex-approach}
%\includegraphics[width=.8\columnwidth]{figures/RRL_Architecture.png}
%\caption{Architecture of the READEX Runtime Library (RRL)}
%\label{fig:rrl}
%\end{figure}

\subsection{Scenario Detection}\label{scenario-detection}
Runtime detection of the upcoming scenario involves several steps. First, the ATM is loaded at the beginning of the application execution. When a new application region is entered during the execution, RRL receives a notification of a region enter event from Score-P. The RRL performs a check to detect if the encountered region is significant by searching for the region in the ATM. If the region is found, it is marked as a significant region. Otherwise, it is marked as an unknown region. Another check that determines if the region must be tuned is performed by computing the granularity of the current region. The region will be tuned only if the granularity is above 100ms. Once the region has been determined to be both significant and coarse-granular enough, additional identifiers that are used to identify the current rts are requested. After the required identifiers are gathered, the current rts is identified by both the call stack and the additional identifiers. Finally, a new configuration for the current rts is applied in order to perform the configuration switching.
For an coarse-granular region marked as unknown, the calibration mechanism is invoked.

\subsection{Configuration Switching}\label{config-switching}
RRL performs the switching of the tuning parameter settings through the parameter control plugins. During the scenario detection step, a new configuration is applied to the current region only if the region entered is found to be significant and coarse-granular. The new configuration is passed to the Parameter Controller in the RRL, which sets the configuration through the respective Parameter Control Plugins (PCP).

Before the current region exits, the RTS Handler receives the notification  from Score-P through Control Center. The RTS Hander checks if the current region was set up for calibration. If yes, it requests the configuration for the currently exited region from the
Calibration module. Once the RTS handler gets back the configuration from the calibration module, it passes this configuration to the TMM which stores the new configuration for the respective rts. If the region was not set up for calibration, then the Parameter Controller is informed that it might want to unset the current configuration. 

The Parameter Controller supports two different modes: \textit{reset} and \textit{no-reset}. The first mode maintains a configuration stack. Whenever a new configuration is set, the previous configuration is pushed onto this stack. When the corresponding unset occurs, the element is removed from the stack and the previous configuration is set. If the no-reset mode is selected, the current configuration stays active until a new configuration is set. The unset is ignored. This behaviour is configurable by the user. By default, \textit{reset} mode is enabled.

\subsection{Calibration}\label{calibr}
 For the already seen rts's, RRL extracts the optimal configuration from the ATM. For the un-seen ones, the RRL calibration mechanism, explained in Section~\ref{sec:calibration}, is used to find the optimal system configuration based on machine learning algorithms and the data stored in the tuning model. 

 