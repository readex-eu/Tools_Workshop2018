\section{Tuning Model Generation} \label{tm-generation}

At the end of the Design Time Analysis (DTA), the Application Tuning Model (ATM) 
is generated based on the best system configuration computed via the tuning 
plugins. A best configuration is found for each runtime situation (rts). One 
option at runtime could have been to switch configuration each time a new rts is 
encountered. This would result in very frequent switching, however, with a 
corresponding switching overhead (time and energy). To avoid this, the READEX 
tool-suite merges rts's into scenarios, and assigns a common configuration for 
all rts's in that scenario. An obvious solution is to merge rts's with identical 
best found configurations into scenarios. Since the set of possible 
configurations is very large, this can still result in too frequent switching. 
The READEX tool-suite hence also merges rts's with similar configurations into 
scenarios and uses one common configuration for all rts's in a scenario. More 
details of this process will be given below.

The ATM thus consists of the complete set of rts's, each determined by their 
identifier values and each with a link to the corresponding scenario. In 
addition, the ATM contains a list of the scenarios, each with a corresponding 
configuration. The ATM is given as a readable JSON text file. By avoiding to 
include the complete configuration for each rts, the ATM size is reduced. 

To cluster rts's that should be merged into a scenario, a three step process is 
used: dendogram generation, cluster generation, and scenario creation. The 
dendogram is a tree expressing the (dis-)similarity of rts's based on their 
configurations. The tree's leaves represent the rts's, whereas all intermediate 
nodes represent the clusters of these rts's.  Lance-Williams algorithm 
\cite{Lance1967} recursively computes the inter-cluster distance, which is used 
as a metric to always merge the closest clusters. The algorithm continues until 
all clusters are merged into a single cluster. 

The second step consists in making a cut through the tree in such a way that it 
minimizes the dispersion of configurations within a cluster, while at the same 
time maximizing the dispersion between clusters. Using the 
\emph{Calinski-Harabasz Index} \cite{Calinski1974}, these values can be combined 
in such a way that an optimal number of clusters can be found. This can again be 
used to compute the point where the tree is cut to create clusters. Each cluster 
represents a scenario.

The final step is to select a configuration for each scenario, based on the 
configurations of the rts's belonging to each of them. Currently, this can be 
done either by picking one of the configurations at random, or by computing the 
average for each configuration parameter. Picking the random configuration is 
the current default, since the configurations are in any case close.

