\subsection{Input Identifiers} \label{sec:input}

DTA exploits the characteristic variations in the applications for different input sets to improve the tuning model by identifying more rts's due to variations in application executions. Different application executions are identified via input identifiers in the form of key-value pairs in a separate file. For example, in the MG benchmark, the maximum grid resolution may change the compute intensity characteristics on the different grid levels. The change from memory to compute bound on coarser grids might happen later if the resolution of the finest gird is higher. To be more precise, the combination of the finest resolution and the number of processes onto which the grids are distributed influence the computational characteristics. As the number of processes increase, data distributes better over the caches which results switching between memory bound and compute bound. Hence, the numbers of processes may also be considered as input identifiers. 

Each input identifier is attached to a specific ATM while generating the tuning model at the end of the \texttt{readex\_intraphase} plugin. In order for all of the tuning information from the different ATMs to be usable by the RRL, these tuning models must be merged into a single tuning model. This is performed by the tuning model merger.

The tuning model merger is a standalone program that takes all the ATMs as input on its command line and outputs a new ATM incorporating all the rts's from the input ATMs. The program does this by first deserializing all ATMs. It extracts all tuning information from the ATMs, such as rts's and their system configurations as well as the corresponding input identifiers. The scenarios from the individual ATMs are discarded. Next, the tuning model merger filters all rts's in order to avoid duplicated rts's in the final tuning model. The next step is to produce a new set of scenarios by clustering all rts's as described in Section \ref{tm-generation}. Finally, the tuning model merger serializes the merged tuning model information and outputs the new ATM in the JSON format.
