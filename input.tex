\subsection{Input Identifiers} \label{sec:input}
\textbf{TUM and NTNU (1 page)}
\textcolor{blue}{
\begin{enumerate}
	\item Motivation
	\item Explain how input identifiers are given in separate file.
	\item Talk about handling of processes/threads
	\item Tuning model merging
\end{enumerate}}
DTA is executed to observe the characteristic variations for different input sets. This helps the user to improve the tuning model by identifiying more rts's due to variations in application executions. The application executions are identified via input identifiers in the form of key-value pairs in a separate file.

During DTA a separate ATMs is generated for each specific inpput set. In order for all of the tuning information from the different ATMs to be usable by the RRL, these tuning models must be merged into a single tuning model. This is the task of the tuning model merger.

The tuning model merger is a standalone program that takes all the ATMs as input on its command line and outputs a new ATM incorporating all the rts's from the input ATMs. The program does this by first deserializing all ATMs. It extracts all tuning information from the ATMs, such as rts's and their system configurations as well as the corresponding input identifiers. The scenarios from the individual ATMs are discarded. Next the tuning model merger filters all rts's in order to avoid duplicated rts's in the final tuning model. The next step is to produce a new set of scenarios by clustering all rts's as described in Section \ref{tm-generation}. Finally, the tuning model merger serializes the merged tuning model information and outputs the new ATM in the JSON format.
